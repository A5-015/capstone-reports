\section{Problem Definition}
\subsection{Problem Analysis}
In order to clarify the scope of the problem, the following questions were considered:

\begin{enumerate}

    \item Who has the problem?
    \begin{itemize}
        \item Researchers and developers who are working with clusters of mobile machines (drones, cars, rovers, etc.) operating at the internet edge with embedded systems that are tasked to achieve a mission autonomously.
    \end{itemize}
    
    \item What is the problem?
    \begin{itemize}
        \item In certain cases, these autonomous devices may fail to complete a mission or a set of tasks due to communication issues amongst themselves when using a hub-spoke system. There is no flexible and open-source software library to remedy this.
    \end{itemize}
    
    \item Where does the problem occur?
    \begin{itemize}
        \item The problem can occur in geographies where stable communication between devices connection is not possible or where the devices are not easily accessible to service by humans.
    \end{itemize}
    
    \item When does the problem occur?
    \begin{itemize}
        \item The problem can occur at any moment since it may be a result of the leader or a hub node failing within a hub-spoke topology.
    \end{itemize}
    
    \item Why does the problem occur?
    \begin{itemize}
        \item The problem occurs because all traffic is routed through these leader or hub nodes. If this node fails, communication also fails because other nodes are not able to organize themselves. The existing solutions are consensus algorithms modified by researchers but unavailable for easy deployment to the wider population.
    \end{itemize}
\end{enumerate}

We also derive further questions from the analysis above:

\begin{enumerate}

    \item What has the typical focus been on addressing the problem?
    \begin{itemize}
        \item In our background research, we have determined that most researchers attempt to develop modified consensus algorithms to better fit the mesh network; however, these modified frameworks are often only discussed in theory, and the implementation never published in an easy to deploy format.
    \end{itemize}
    
    \item How do we address the issue of developing a ubiquitous solution accessible to multiple hardware devices?
    \begin{itemize}
        \item While the scope of the deliverables in this project may be limited to specific hardware, we can attempt to develop a community around the idea to encourage contributions from other researchers. To this effect, we can also select a popular piece of hardware to allow for more testers and wider outreach.
    \end{itemize}
\end{enumerate}



\subsection{Problem Clarification}
%   Problem Clarification
    % Black-box model
%%%%%%% %%%%%%% %%%%%%% 

Autonomous devices with embedded systems and limited power and computational resources can benefit from consensus algorithms for coordination tasks. However, such an implementation is not yet openly available. To address this, in this section, we attempt to explore the system and the sub-problems it implies.

\begin{figure}[H]
    \centering
    \includegraphics[width=0.90\columnwidth]{final-proposal/images/blackbox.png}
    \caption{A black box diagram for each node of the system}
    \label{fig:black_box}
% graph LR
%   cons(Consensus Algorithm)
%   netProc(Networking Protocol)
%   netTop(Network Topology)
%   netLib(Networking Library)
%   cont(Microcontroller)
%   data(Data)
%   groupaction(Group Action)
%   localaction(Individual Action)

%   data --> cons
%   subgraph " "
%     cons --> netLib --> cont --> netProc --> netTop
%   end
%   netTop --> groupaction
%   cont -..-> localaction
\end{figure}

The black-box model of a node in the described system, broken into its sub-problems, is shown in Figure \ref{fig:black_box}. Each of these sub-problems is further discussed in the following sub-sections. The sensors on a node will be collecting data, which will then be fed into the decision logic, implemented as a consensus algorithm. Each of these nodes will be communicating with one another using a Wi-Fi-based mesh network and updating their logs as a result of this communication. Eventually, each of the nodes will then carry out some actuation or state indication.

As there are multiple implementations of an 802.11 based mesh network, it is out of the scope of this project to be able to interface with all of them. Our implementation will focus on what is readily available and will be designed in a way so that with minimal future development another implementation of a mesh network could replace the one we use.

There are also quite a few options for consensus algorithms. However, our system will be limited to a single consensus algorithm, namely Raft \cite{raft_paper}. Section \ref{conceptualization_consensus} describes the reasoning behind selecting Raft.


\subsubsection{Networking Protocols}
There are a plethora of network types for wireless communication, often dependent on the application of the system. From literature, we identify four dominant protocols: Bluetooth, Ultra-wideband (UWB), Zigbee, LoRa, and Wifi \cite{lee2007comparative, ti_lethaby2017wireless}.

Mesh networks are not limited to any one of these standards. In fact, "IoT-related wireless technologies [...] are extremely heterogeneous in terms of protocols, performance, reliability, latency, cost-effectiveness, and coverage",  \cite{iot_survey_cilfone2019wireless}. Each standard has its niche of operation: IoT devices may employ the IEEE 802.15.4 standard or Bluetooth for short-range communications, whereas mid to long-range communications may make use of IEEE 802.11, which was amended in 2011 to support mesh networks \cite{ti_lethaby2017wireless, iot_survey_cilfone2019wireless}. Figure \ref{fig:comparison_protocols} shows a comparative performance analysis of LoRa, Zigbee, WiFi, and Bluetooth, highlighting how each protocol has its strengths and weaknesses.


\begin{figure}[H]
    \centering
    \includegraphics[width=0.65\columnwidth]{final-proposal/images/comp_perf_analysis.png}
    \caption{Comparative performance analysis of four protocols. \\ Adapted from \cite{iot_survey_cilfone2019wireless}}
    \label{fig:comparison_protocols}
\end{figure}


While the implementation of mesh networks and consensus algorithms are agnostic to the protocol underneath, it is important to address these concepts in the context of real-world constraints. Current IoT research "assumes that the devices are equipped with low-power IEEE 802.15.4 (Zigbee) transceivers" \cite{disney_glaropoulos2013enhanced}. Yet, a significant portion of deployed embedded systems and consumer electronics carry IEEE 802.11 compatible hardware \cite{disney_glaropoulos2013enhanced}. 
The "wide penetration of IEEE 802.11" \cite{disney_glaropoulos2013enhanced} then makes it the more attractive choice because the hardware does not need to be redeployed to support newer standards. Despite 802.11's notoriety as high power consumption, research indicates that additional firmware controlling the sleep schedule can help lead to significant energy savings \cite{disney_glaropoulos2013enhanced, barghi2019practicalpower}

Furthermore, 802.11 has built-in support for mesh networks as a result of the 802.11s amendment. The amendment describes mesh stations (mesh STAs) and their ability to link with one another without the requirement of a central AP; rather, certain nodes may act as mesh access points (MAP) to connect to another network \cite{iov_wu2016internet, optical_zeitgeist_laboratory_2011}.

Given 802.11's ubiquity and built-in support for mesh networks, we found it fit to work with an 802.11-based network.

\subsubsection{Consensus Algorithms}
\label{conceptualization_consensus}
With the rapid increase in adoption and development of distributed \& multi-agent systems, achieving a consensus across these systems became an important issue because of the high potential for scalability \cite{Ge_Han_Ding_Zhang_Ning_2018}. Achieving consensus across such systems allows autonomous air vehicles, cooperative IoT devices, sensor networks to be built at a scale. 

The distributed consensus algorithms can be classified into three categories based on their hierarchical structure: \emph{leaderless consensus algorithms}, \emph{leader-following consensus algorithms}, and \emph{containment control algorithms} \cite{consensus_systems_survey}. In \emph{leaderless consensus algorithms}, there are no leaders, and the nodes within the system are expected to asynchronously converge on a target \cite{Ge_Han_2017}. However, in \emph{leader-following consensus algorithms}, there is a leader node that orchestrates the actions of the rest of the network. Finally, in \emph{containment control algorithms}, there might be more than one leader to orchestrate different types of operations within the system. A visualization of the different types of consensus algorithms can be seen in Figure \ref{fig:consensus_types_detailed}, where green circles are the leaders in the system, and orange lines represent the logical connections between the consensus members.


\begin{figure}[H]
    \centering
    \includegraphics[width=0.7\columnwidth]{final-proposal/images/consensus_types.png}
    \caption{Left: A leaderless consensus algorithm, Middle: A leader-following consensus algorithm, Right: A containment control algorithm}
    \label{fig:consensus_types_detailed}
\end{figure}


The distributed consensus algorithms can also be classified into two categories based on when they transfer information to reach a consensus: \emph{time-triggered consensus algorithms} and \emph{event-triggered consensus algorithms} \cite{consensus_systems_survey}. In \emph{time-triggered consensus algorithms}, the nodes in the system initiate the information transfer once a set amount of time elapses. Additionally, if the consensus algorithm is asynchronous by design, each node can have a randomly set time duration before they send information. On the other hand, in \emph{event-triggered consensus algorithms}, the nodes in the system initiate information transfer when an external event occurs. A visualization of the different types of consensus algorithms can be seen in Figure \ref{fig:consensus_time_based_and_evevent_based}, where orange circles represent the timer of the nodes, green arrow represents the external event trigger, and red arrows represent the information transfer.

\begin{figure}[H]
    \centering
    \includegraphics[width=0.4\columnwidth]{final-proposal/images/consensus_time_based_and_evevent_based.png}
    \caption{Left: A time-triggered consensus algorithm exchanging messages when the timer goes off, Right: An event-triggered consensus algorithm exchanging messages when an event occurs (green arrow)}
    \label{fig:consensus_time_based_and_evevent_based}
\end{figure}

Paxos and Raft are among the most popular \& practical distributed \emph{time-triggered} and \emph{leader-following} consensus algorithms available \cite{paxos_vs_raft}. Paxos has been the industry standard since it was first published, despite its status as a difficult to understand and implement as an algorithm. Since the Raft consensus algorithm aims to be simple and easy to understand \cite{raft_paper}, it is gaining popularity in the industry. Furthermore, the Raft consensus algorithm aims to abstract away how the underlying technology works and avoid a system-specific implementation \cite{paxos_vs_raft}. As a result, the Raft consensus algorithm is more suitable for experimenting and implementing non-traditional topologies like mesh networks.

\subsubsection{Microcontrollers}
Microcontroller units (MCUs) are a hefty consideration for any IoT solution. Unsurprisingly, there is a large variety of MCUs at different levels of "memory, power, computational capability, architecture etc." \cite{bansal2020iotsurveydevices} for various applications. These devices sit at different levels of the IoT ecosystem, implying that not all devices are required to be homogeneous in their capabilities. For discussion purposes, we adopt the classification system described in Bansal & Kumar's survey of IoT technologies \cite{bansal2020iotsurveydevices}. Class 0 devices may have 1-50KB of RAM with clock speeds below 100MHz, class 1 devices range from 100KB-100MB of RAM with clock speeds between 100MHz-1.5Ghz, and class 2 devices generally have resources greater than those of class 1 \cite{bansal2020iotsurveydevices}. Typically, class 0 devices such as the TELOSB are employed for Wireless Sensor Network (WSN) use-cases to collect data and perhaps perform simple actuation. On the other hand, class 1 devices, such as the well-known Raspberry Pis, are more suitable for coordination, media processing, data filtering tasks, etc.

Given our design constraints, namely dynamic topology, and decision-making capabilities, we would expect this system to work with embedded systems on mobile technologies such as drones, vehicles, and rovers. While such devices are often fitted with class 2 devices, the goal of our project is to be able to deploy such a networking ability on class 1 devices as well. Espressif's ESP8266 chip stands out as an option: the chip is designed to be a WiFi chip with a focus on low-cost and minimal peripherals. The MCU operates between 80-160MHz and has 50KB of SRAM with up to 16MB of external flash memory \cite{espressif:esp8266}, so it is on the lower end of the class 1 category. Moreover, there is a decent open-source community dedicated to the MCU, and Espressif has released a mesh networking software development kit (SDK) for its chips with detailed documentation \cite{esp-mesh-docs}. 


\subsection{Problem Statement}
%   Problem Statement
    % Clarify the scope of the problem [kinda done]
    % Project Aim [done]
%%%%%%% %%%%%%% %%%%%%% 

We have identified the problem to be the lack of a flexible and easy to access software library that implements a consensus algorithm atop a mesh network for low-power embedded systems in dynamic environments.

% The objective of this Capstone Project is to design and develop a resilient wireless mesh network architecture coupled with a consensus algorithm for low-power embedded devices in dynamic topologies, demonstrated using 802.11-based ESP8266 boards.

In order to address this problem, we aim to create an open-source software library that can be incorporated into an application that requires nodes to work autonomously together. We plan to demonstrate our software library through prototype boards that will each consist of an ESP8266 MCU, power sensors, and LEDs. Our software library will be installed on these boards, enabling them to form a mesh network and run the Raft consensus algorithm. We will show how a signal propagates in the Raft consensus algorithm within the mesh network using LEDs, as shown in Figure \ref{fig:mesh_signal_propagation}. We will also demonstrate how the network can realign \& recover by turning on and off randomly selected ESP8266 microcontrollers to emulate a dynamic environment.

\begin{figure}[H]
    \centering
    \includegraphics[width=0.7\columnwidth]{final-proposal/images/mesh_signal_propagation.png}
    \caption{An ESP8266 cluster transitioning to the blue state from the red state using our software library. Red and blue LEDs are used to represent the states.}
    \label{fig:mesh_signal_propagation}
\end{figure}

% giving specs and constraints in quantitative terms

Furthermore, with our software library, we aim to achieve a packet delivery rate of a minimum 95\% and an average latency smaller than 20ms between nodes on the mesh network while having a maximum leader selection time of 1000ms for the Raft consensus algorithm. The software library should also be able to accommodate at least 500 nodes in the mesh network and have a total size of 2MB when compiled with the avr-gcc compiler.



