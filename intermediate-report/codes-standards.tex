\section{Codes and Standards}

This project relies on two communication standards: the IEEE 802.11 wireless standard for inter-node communication and the I2C protocol for communication between integrated chips on the demonstration PCB.

 Specifically, the ESP8266 chip supports the IEEE 802.11B/G/N protocols \cite{espressif:esp8266}. The mesh networking library used in this project, painlessMesh, only supports IEEE 802.11B and IEEE 802.11G, with the latter being the default protocol \cite{painlessMesh:wifi_standard}. Using the IEEE 802.11G protocol on the ESP8266 requires a transmit power of approximately $\SI{17}{dBm}$ and communication occurs within the $\SI{2.4 - 2.5}{GHz}$ frequency range \cite{espressif:esp8266}. Section \ref{section:networking_protocols} further discusses the IEEE 802.11 protocol in comparison to other available communication protocols.

The demonstration PCB uses the I2C protocol for communication between the microprocessor, the sensors, and the display modules. The I2C protocol, developed by Philips Semiconductors, transmits data over a single SDA line and clocks the device communication over the SCL line. A single I2C bus can theoretically communicate with up to 127 secondary devices, although this number changes in practice based on the data rate and communication distance \cite{i2c_overview}. The I2C protocol satisfies the integrated chip communication requirements for the demonstration PCB.