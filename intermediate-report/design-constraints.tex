\section{Design Constraints}
\subsection{Technical}

Taking into account market demands, the history of the topic, and potential applications, there are various technical constraints on our system:

\begin{itemize}
    \item \textit{Resilient to threats:}
    \begin{itemize}
        \item  The system should also be resilient to handling abrupt changes to the system, such as the loss of a leader, ensuring that the task is truly distributed. 
        \item The final design aims to recover from the loss of a consensus leader within 1000\si{\ms}.
    \end{itemize}
    
    \item \textit{Scalability:} 
    \begin{itemize}
        \item The system should be able to perform with various densities of nodes. For example, increasing the number of nodes in a given area should not significantly increase latency. 
        \item The final design aims to support at least 500 nodes in an area covering 1000\si{m^2}.
    \end{itemize}
    
    \item \textit{Low power consumption:}
    \begin{itemize}
        \item Since this system is centered around embedded devices deployed in the real-world, the implementation should consider limitations in access to power. 
        \item The final design is aimed to be consuming below 100\si{mAh} on average use.
    \end{itemize}
    
    \item \textit{Small footprint:} 
    \begin{itemize}
        \item The compiled code for the system should be small enough to fit into the memory of an embedded system. 
        \item The final design is aimed to fit into an embedded 2\si{MB} flash memory.
    \end{itemize}
\end{itemize}

\subsection{Non-Technical}

We also consider non-technical constraints:

\begin{itemize}
    \item \textit{Modular design:} 
    \begin{itemize}
        \item  Although the implementation in this project is for an IEEE 802.11 based chip, the system should be designed so that it may be ported to another underlying protocol.
    \end{itemize}
    
    % \item \textit{Decision making:} 
    % \begin{itemize}
    %     \item  The system should be capable of coordinating all of the nodes to successfully arrive at decisions, ensuring that the consensus algorithm is meaningfully implemented.
    % \end{itemize}
    
    \item \textit{Dynamic topology:}
    \begin{itemize}
        \item  The system should be capable of performing on mobile technologies, where nodes are constantly entering and exiting the network.
        %, potentially at high speeds.
    \end{itemize}
    
    \item \textit{Versatility:}
    \begin{itemize}
        \item  Given that neither consensus algorithms nor mesh topologies are concepts with limited applications, the system should maintain this quality and ideally be able to operate on various technologies such as UAVs, cars, rovers, etc.
    \end{itemize}
    
    \item \textit{Open-Source:}
    \begin{itemize}
        \item  This project should be built with the potential to have a community develop around it for future development, implying a permissible license, proper documentation, and ease of access. The system should not be limited to expensive and obscure technology.
    \end{itemize}
\end{itemize}
 

