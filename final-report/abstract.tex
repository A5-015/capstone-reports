\section{Abstract}
% Used to indicate where a new paragraph starts within the abstract 
\newcommand{\nextInternalParagraphStartsHere}{}

Consensus is a fundamental problem in fault-tolerant distributed systems and involves multiple nodes arriving at a coordinated decision. Reaching a consensus is more challenging if the system is dynamic and if it makes decisions in real-time, as is the case with autonomous vehicles and mobile sensor networks. Thus, consensus algorithms ensure that a cluster of devices can cooperatively complete its tasks even if the cluster loses its leader. However, if a consensus algorithm is built upon a typical hub-spoke network topology, the algorithm may be rendered useless if the singular network access point fails. \nextInternalParagraphStartsHere A mesh network is an alternative, non-hierarchical topology for local networks in which devices can directly communicate amongst themselves without a central node coordinating the process. As a result, a mesh network is resilient to a single point of failure. \nextInternalParagraphStartsHere Our capstone project implements Raft, a distributed consensus algorithm, atop a mesh network for use in low-power embedded systems. As a part of the capstone project, an open-source software library was developed and prototyped on custom-designed printed circuit boards (PCB) with an ESP8266 chip. Our compiled software library consumes 0.33MB of flash memory and is capable of supporting over 100 nodes. During elections, it takes an average of 300ms to elect a new leader with insignificant variation as the network size grows.
