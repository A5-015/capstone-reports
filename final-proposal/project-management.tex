\section{Project Management}
\subsection{Work Breakdown Structure}
The project was broken down into smaller tasks and their duration was estimated in days. Later, the main tasks were divided into smaller sub-tasks and they were placed into a \emph{work breakdown structure} as shown in Table \ref{tab:work_breakdown_structure}. The \emph{work breakdown structure} ensured us to stay on track in our project and slow down or speed up when required to complete the project successfully.

\begin{table}[ht]
    \scriptsize
    
    % Set row height
    \renewcommand{\arraystretch}{1.125}
    
    %%%%%%%%%%%%%%%%%%%%%%%%%%%%%%%%%%
    %%%%%%%% HELPER FUNCTIONS %%%%%%%% 
    %%%%%%%%%%%%%%%%%%%%%%%%%%%%%%%%%%
    
    % Set current date [YYYY-MM-DD]
    \newcommand{\setCurrDate}[3]{\setdatenumber{#1}{#2}{#3}}
    
    % Custom date format
    \def\datedate{\thedatemonth/\thedateday/\thedateyear}
    %\def\datedate{\thedateday-\thedatemonth-\thedateyear}
    
    % Takes number of days as an argument and prints "arg1 & DATE & DATE+arg1"
    \newcommand{\TEdate}[1]{
        \setdatebynumber{\thedatenumber}
        \multicolumn{1}{c}{#1} & 
        \datedate & 
        \addtocounter{datenumber}{#1} \setdatebynumber{\thedatenumber}
        \datedate
    }
    
    % Stuff for numbering table items
    \newcounter{TableEntryID}
    \newcounter{SubTableEntryID}
    \setcounter{SubTableEntryID}{0}
    \setcounter{TableEntryID}{0}
    \newcommand\showTE{\setcounter{SubTableEntryID}{0}\stepcounter{TableEntryID}\theTableEntryID.\theSubTableEntryID \ }
    \newcommand\showSubTE{\stepcounter{SubTableEntryID}\theTableEntryID.\theSubTableEntryID \ }
    
    % Stuff for automating table rows
    \newcommand{\tableEntry}[1]{\hline \multirow{2}{*}{\showTE #1}}
    \newcommand{\subTableEntry}[2]{& \showSubTE #1 & \TEdate{#2} \\ \cline{2-5}}
    \newcommand{\initialTableEntry}[2]{&0.1 #1 & \TEdate{#2} \\ \cline{2-5}}
    \newcommand{\finalTableEntry}[2]{\hline &\showTE #1 & \TEdate{#2} \\ \hline}

    \captionof{table}{Work breakdown structure of the project}
    \label{tab:work_breakdown_structure}
    
    \begin{center}
        \begin{tabular}{|l|p{30em}|p{3.5em}|r|r|}
            % Table HEAD
            \hline
            \multicolumn{2}{|l|}{\multirow{3}{9cm}{\textbf{ramen: Design and Development of a Raft Consensus Algorithm Coupled With a IEEE 802.11 Based Mesh Network for Embedded Systems}}} & \multicolumn{3}{c|}{Dates and Duration} \\ \cline{3-5}
            \multicolumn{2}{|l|}{ } & Duration (Days) & \multicolumn{2}{c|}{Planned Dates} \\ \cline{3-5}
            \multicolumn{2}{|l|}{ } & & \multicolumn{1}{c}{Start} & \multicolumn{1}{|c|}{End} \\ \hline
            
            %%%%%%%%%%%%%%%%%%%%%%%%%%%%%%%%%%%%%%%%%%%%%%%
            %%%%%%%% ACTUAL TABLE DATA STARTS HERE %%%%%%%% 
            %%%%%%%%%%%%%%%%%%%%%%%%%%%%%%%%%%%%%%%%%%%%%%%
            
            \setCurrDate{2020}{09}{06}
            \initialTableEntry{Begin Project}{1}
            
        	\tableEntry{Background Research}
        	    \subTableEntry{Research on existing problems in IoT devices}{3}
            	\subTableEntry{Research on existing problems in embedded devices}{3}
            	\subTableEntry{Research on currently existing solutions}{3}
            	\subTableEntry{Identifying technical \& non-technical constraints}{4}
            	\subTableEntry{Revise problem statement}{1}
        	
        	\tableEntry{Generate Concepts}
        	    \subTableEntry{Functionality decompose of the project}{1}
            	\subTableEntry{Research on literature for similar solutions}{5}
             	\subTableEntry{Experimenting with existing consensus and mesh networking protocols protocols}{3}
             	\subTableEntry{Consensus protocol, network topology, and  microprocessors selection}{2}
            	
        	\tableEntry{Begin Detailed Design}
        	    \subTableEntry{Perform detailed analysis of the concepts}{2}
        	    \subTableEntry{Perform detailed analysis of the available components}{5}
        	    \subTableEntry{Select components}{2}
            	\subTableEntry{Perform simulations with consensus algorithms}{5}
            	\subTableEntry{Perform simulations with WiFi chips}{5}
            	\subTableEntry{Create workflow diagrams for the code}{2}
        	
        	\tableEntry{Build Prototype}
        	    \subTableEntry{Create a GitHub repository}{1}
        	    \subTableEntry{Code a mesh network for ESP8266 chip}{20}
        	    \subTableEntry{Adapt the Raft consensus algorithm for mesh networks}{25}
        	    \subTableEntry{Code an adapter for our Raft implementation for ESP8266 chip specifically}{25}
        	    \subTableEntry{Ensure the functionality of connection between the Raft consensus algorithm implementation and the mesh networking adapter}{5}
        	    \subTableEntry{Write tests for the code}{5}
            	\subTableEntry{CAD drawings for the enclosure}{4}
            	\subTableEntry{Design the PCB}{10}
            	\subTableEntry{Order the PCB for fabrication}{2}
            	\subTableEntry{Purchase the PCB components}{5}
            	\subTableEntry{Assemble the PCB}{3}
            	\subTableEntry{3D printing the enclosure}{2}
            	\subTableEntry{Assembling the prototype}{3}
            	\subTableEntry{Upload the code to the prototype}{2}
            	\subTableEntry{Check the functionality of the prototype and the code}{5}
        	
        	\tableEntry{Test Prototype}
            	\subTableEntry{Develop testing protocol}{7}
            	\subTableEntry{Perform tests}{7}
        	
        	\tableEntry{Documentation and Reporting}
        	    \subTableEntry{Generating documentation from the code}{2}
            	\setCurrDate{2020}{10}{01}\subTableEntry{Preparation of the Intermediary Report I}{10}
            	\setCurrDate{2020}{11}{20}\subTableEntry{Preparation of the Final Proposal}{10}
            	\subTableEntry{Preparation of the Proposal Presentation}{8}
            	\setCurrDate{2020}{3}{5}\subTableEntry{Preparation of the Intermediary Report II}{10}
            	\setCurrDate{2020}{4}{20}\subTableEntry{Preparation of the Final Poster}{8}
            	\subTableEntry{Preparation of the Final Presentation}{10}
        	
        	\finalTableEntry{End Project}{1}
        	
        \end{tabular}
    \end{center}
\end{table}


\newpage
\subsection{Design Structure Matrix}
After creating the \emph{work breakdown structure}, the \emph{design structure matrix} shown in Table \ref{tab:design_structure_matrix} was created. The \emph{design structure matrix} helped us to order the task to streamline our project tasks.


\begin{table}[ht]
    \scriptsize
    \centering
    \renewcommand{\arraystretch}{1.3}
    \caption{Design structure matrix of the project}
    \label{tab:design_structure_matrix}
    \begin{tabular}{r|c|c|c|c|c|c|c|c|c|c|c|c|c|c|c|}
        \cline{2-16}
                                        &   & A & B & C & D & E & F & G & H & I & J & K & L & M & N \\ \cline{2-16}
         Begin Project                  & A & A &   &   &   &   &   &   &   &   &   &   &   &   &   \\ \cline{2-16}
         Background Research            & B & X & B &   &   &   &   &   &   &   &   &   &   &   &   \\ \cline{2-16}
         Consensus Algorithm Selection  & C &   & X & C &   &   &   &   &   &   &   &   &   &   &   \\ \cline{2-16}
         Network Topology Selection     & D &   & X &   & D &   &   &   &   &   &   &   &   &   &   \\ \cline{2-16}
         Concept Generation             & E &   &   & X & X & E &   &   &   &   &   &   &   &   &   \\ \cline{2-16}
         Detailed Design                & F &   &   & X & X & X & F &   &   &   &   &   &   &   &   \\ \cline{2-16}
         Simulation                     & G &   &   &   &   &   & X & G &   &   &   &   &   &   &   \\ \cline{2-16}
         Finalize Design                & H &   &   &   &   &   & X & X & H &   &   &   &   &   &   \\ \cline{2-16}
         Coding the software            & I &   &   & X & X & X &   & X & X & I &   &   &   &   &   \\ \cline{2-16}
         CAD Drawings                   & J &   &   &   &   &   &   & X & X &   & J &   &   &   &   \\ \cline{2-16}
         Purchase Components            & K &   &   &   &   &   &   &   & X &   & X & K &   &   &   \\ \cline{2-16}
         Manufacture Components         & L &   &   &   &   &   &   &   &   &   & X & X & L &   &   \\ \cline{2-16}
         Assembly and Testing           & M &   &   &   &   &   &   &   & X & X &   &   & X & M &   \\ \cline{2-16}
         Finish Project                 & N &   &   &   &   &   &   &   &   &   &   &   &   & X & N \\
        \cline{2-16}
    \end{tabular}
\end{table}


\subsection{Critical Path}
The \emph{critical path method} was used to identify the bottlenecks in the project. The duration for each project component were calculated in days and placed into the \emph{critical path} graph shown in Figure \ref{fig:critical_path}. After our analysis, we determined that the critical path for our project is implementing the Raft consensus algorithm on top of a mesh network. We have to have a working Raft coupled with mesh implementation in order to perform tests. For our project, building a hardware prototype doesn't fully depend on the software implementation and the software development do not require a specific hardware.

\begin{figure}[ht]
    \centering
    
    \resizebox{0.80\pdfpagewidth}{!}{
        \begin{tikzpicture}
        
            %%%%%%%%%%%%%%%%%%%%%%%%%%%%%%%%%%
            %%%%%%%% HELPER FUNCTIONS %%%%%%%% 
            %%%%%%%%%%%%%%%%%%%%%%%%%%%%%%%%%%
            
            % Custom color
            \definecolor{airforceblue}{rgb}{0.36, 0.54, 0.66}
            
            % Function for generating nodes in the graph
            \newcommand{\pathNode}[5][black]{
                \scriptsize
                \node(#2)[shape=rectangle][#3] {
                    \begin{tcolorbox}[
                        rounded corners,
                        colback=airforceblue!35,
                        colframe=#1!80,
                        arc=1.5mm,
                        box align=center,
                        halign=center,
                        valign=center,
                        text width=2cm,
                        left=0.5mm,
                        right=0.5mm,
                        top=0.5mm,
                        bottom=0.5mm,
                        title = {\centering\makebox[\linewidth][c]{\color{white}#5 Days}}
                    ]
                    \color{black}#4
                    \end{tcolorbox} \\
                };
            }
            
            % Function for generating a terminal node (like start % end) in the graph
            \newcommand{\terminalNode}[4][black]{
                \scriptsize
                \node(#2)[circle, draw=#1!80, fill=airforceblue!35, very thick, minimum size=7mm][#3]{#4};
            }
            
            % Function for drawing the arrows
            \newcommand{\pathArrow}[2]{
                \draw[->, very thick] (#1) -- (#2);
            }
        
            %%%%%%%%%%%%%%%%%%%%%%%%%%%%%%%%%%%%%%%%%%%%%%%
            %%%%%%%% ACTUAL GRAPH DATA STARTS HERE %%%%%%%% 
            %%%%%%%%%%%%%%%%%%%%%%%%%%%%%%%%%%%%%%%%%%%%%%%
            
            % Nodes
            \terminalNode[red]{start}{}{Capstone Project}
            
            \pathNode[red]{software_1}{above right = of start, yshift=1cm, xshift=1cm}{Coding a software library}{20}
            \pathNode[red]{software_4}{above right = of software_1, yshift=-1.25cm}{Creating a mesh network}{25}
            \pathNode[red]{software_5}{right = of software_4}{Adapting Raft consensus to mesh the network}{25}
            \pathNode{software_2}{below right = of software_1, yshift=1.25cm, xshift=2cm}{Creating an adapter code for ESP8266 specifically}{20}
            \pathNode[red]{software_3}{below right = of software_5, yshift=1cm}{Testing the code in a simulation}{10}
            
            
            \pathNode{hardware_1}{below right = of start, xshift=1cm}{Building a Hardware Prototype}{15}
            \pathNode{hardware_4}{below right = of hardware_1, yshift=1.25cm}{Prototype enclosure design}{10}
            \pathNode{hardware_2}{above right = of hardware_1, yshift=-1.25cm}{Simulating WiFi radios}{10}
            \pathNode{hardware_3}{right = of hardware_2}{PCB Design}{10}
            \pathNode{hardware_5}{right = of hardware_4}{Manufacturing Components}{15}
            
            
            \pathNode[red]{software_6}{right = of hardware_3, xshift=2.25cm}{Testing on the hardware}{25}
            
            % Arrows
            \pathArrow{start.east}{software_1.west}
            \pathArrow{start.east}{hardware_1.west}
            
            \pathArrow{software_1.east}{software_2.west}
            \pathArrow{software_1.east}{software_2.west}
            \pathArrow{software_1.east}{software_4.west}
            \pathArrow{software_4.east}{software_5.west}
            \pathArrow{software_2.east}{software_3.west}
            \pathArrow{software_5.east}{software_3.west}
            \pathArrow{software_3.south}{software_6.west}
            
            %\pathArrow{software_5.south}{software_2.north}
            
            \pathArrow{hardware_1.east}{hardware_4.west}
            \pathArrow{hardware_1.east}{hardware_2.west}
            \pathArrow{hardware_2.east}{hardware_3.west}
            \pathArrow{hardware_3.south}{hardware_5.north}
            \pathArrow{hardware_4.east}{hardware_5.west}
            \pathArrow{hardware_4.east}{hardware_3.west}
            \pathArrow{hardware_5.east}{software_6.west}
            
            \pathArrow{hardware_2.north}{software_2.west}
    
        \end{tikzpicture}
    }
    \caption{Critical path of the project tasks}
    \label{fig:critical_path}
\end{figure}





\newpage
\subsection{Gantt Chart}
The \emph{Gantt Chart} shown in Figure \ref{fig:gantt} was utilized to visualize the tasks and their duration. The \emph{Gantt Chart} helped us to track our progress on the sub-tasks and overall progress in the project as the time progressed.

\vspace{5mm}

\ganttset{calendar week text = \small {\startday}}

\hvFloat[rotAngle=0]{figure}{
    \resizebox{0.8\pdfpagewidth}{!}{
        \begin{ganttchart}[
            newline shortcut=true,
            bar label node/.append style={align=right},
            time slot format = isodate, 
            vgrid = {*{6}{dotted}, *{1}{dashed}},
            hgrid,x unit=1mm,
            hgrid style/.style={draw=black!5, line width=.75pt},
            time slot format=little-endian,
            linespacing=0.5]
            {01-09-2020}{15-06-2021}
            \gantttitlecalendar{year, month=shortname, week=4}\\
            %%%%%%%%%%%%
            \ganttgroup{Background Research}{07-09-2020}{21-09-2020} \\
                \ganttbar{Research on existing problems}{7-9-2020}{13-9-2020} \\
                \ganttbar{Research on currently existing solutions}{13-9-2020}{16-9-2020} \\
                \ganttbar{Identifying technical \& non-technical constraints}{16-9-2020}{21-9-2020} \\
            %%%%%%%%%%%%
            \ganttgroup{Generate Concepts}{21-09-2020}{02-10-2020} \\
                \ganttbar{Functionality decompose of the project}{21-9-2020}{22-9-2020} \\
                \ganttbar{Research on literature for similar solutions}{22-9-2020}{27-9-2020} \\
                \ganttbar{Experimenting with existing consensus \ganttalignnewline and mesh networking protocols protocols}{27-9-2020}{30-9-2020} \\
                \ganttbar{Consensus protocol, network topology, \ganttalignnewline and  microprocessors selection}{30-9-2020}{02-10-2020} \\
            %%%%%%%%%%%%
            \ganttgroup{Begin Detailed Design}{02-10-2020}{23-10-2020} \\
                \ganttbar{Perform detailed analysis of the concepts}{02-10-2020}{09-10-2020} \\
                \ganttbar{Select components}{09-10-2020}{11-10-2020} \\
                \ganttbar{Perform simulations}{11-10-2020}{23-10-2020} \\
            %%%%%%%%%%%%
            \ganttgroup{Build Prototype}{23-10-2020}{17-02-2021} \\
                \ganttbar{Code a mesh network for ESP8266 chip}{23-10-2020}{13-11-2020} \\
                \ganttbar{Adapt the Raft consensus algorithm for mesh networks}{13-11-2020}{8-12-2020} \\
                \ganttbar{Code an adapter for ESP8266 specifically}{8-12-2020}{12-1-2021} \\
                \ganttbar{PCB design and CAD drawings}{12-1-2021}{28-1-2021} \\
                \ganttbar{Purchase the PCB components}{28-1-2021}{2-2-2021} \\
                \ganttbar{Assembling the prototype}{2-2-2021}{17-2-2021} \\
            %%%%%%%%%%%%
            \ganttgroup{Test Prototype}{17-02-2021}{03-03-2021} \\
                \ganttbar{Develop testing protocol}{17-2-2021}{24-2-2021} \\
                \ganttbar{Perform tests}{24-2-2021}{3-3-2021} \\
            %%%%%%%%%%%%
            \ganttgroup{Documentation and Reporting}{03-03-2021}{08-05-2021} \\
                \ganttbar{Generating documentation from the code}{3-3-2021}{5-3-2021} \\
                \ganttbar{Preparing Reports, Presentations, and Posters}{5-3-2021}{08-05-2021} \\
            %%%%%%%%%%%%
        \end{ganttchart}
    }
}[The Gantt Chart of the project timeline]{The Gantt Chart of the project timeline}{fig:gantt}