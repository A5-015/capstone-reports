\section{Modeling, Simulation and Optimization Plan}
\subsection{Modeling}

% General idea:
    % talk about proposing the modelling we have done

% writing points:
    % generating a virtual space
    % placing nodes randomly into the space 
    % modelling star topology with optimal hub distribution (k-means)
    % modelling mesh topology with 4-8 connection hardware limitation of ESP
    % talk about using python script to output x, y, z parameters (refer to it as \cc{network_generator} script)

\subsection{Simulation}
We have identified, Coracle \cite{Coracle}, a tool to simulate the Raft consensus algorithm on heterogeneous networks. Coracle is a simulation framework written in OCAML designed to evaluate "distributed consensus algorithms in settings that more accurately represent realistic deployments." \cite{howardCoracleEvaluatingConsensus2015}. The framework allows for users to configure nodes, links, and events. 

A node can be defined as a hub, server, client. From experimentation with the framework, we have learned that in the case of a server node, the node acts as a participant and carries out the responsibilities of a typical node, such as voting and log replication. When a node is configured as a hub it foregoes its responsibility as a participant and acts as a router in the network. Finally, a client node [TODO].

Links can either be unidrectional or bidirectional; moreover, multiple links between two nodes can exist. Essentially, this allows for the freedom to simulate whether a pair of nodes has half-duplex or full-duplex communication. Furthemore, each link can be assigned a latency category, small, medium, or large, to affect the package delivery time from source to destination nodes. Since we plan to model our nodes in a virtual space, we can assign link latency categories based on distances between a pair of connected nodes.

Events allow for nodes and links to activate or deactivate at user-defined timestamps. This can simulate unstable networks with multiple unreliable links that periodically fail. Morevoer, considering real-world networks, this parameter can also be used to simulate the addition of new nodes into the network, with the only limitation being that node links would be predetermined. 

A limitation in modeling and simulation workflow is the lack of traveling nodes, which rebuild their links based on available nearby nodes. We expect to account for this limitation in our physical implementation of the project, where we will be able to test how Raft on mesh behaves with dynamic nodes.

Since Coracle accepts a .JSON file with a specific structure, we expect to write a \cc{simulation\_configuration} script to process the output of the \cc{network\_generator} script and generate a .JSON file for Coracle. Finally, we expect to write \cc{run\_coracle} script to feed the .JSON file into Coracle, run multiple simulations, generate statistics, and display the results. 

For our project, we will model networks of 20, 50, 100, 200, 500, and 1000 nodes in areas of 500$m^2$, 1000$m^2$, and 2000$m^2$, for both star and mesh topologies. For each combination, we will run 1000 simulations and gather and study the generated statistics.

\subsection{Experimental Plan}

