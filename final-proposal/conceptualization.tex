\section{Conceptualization}
\subsection{Concept Generation}

Following the black box model in Figure \ref{fig:black_box}, we generated the concepts in the morphological chart shown in Table \ref{tab:morph_chart}.

\begin{itemize}
	\item \textbf{Networking Protocols} - While we will design our system to be modular and potentially connect with any networking protocol, we must select a protocol upon which to prototype. Through literature search, we were able to find the most commonly used networking protocols available for IoT solutions.
	\item \textbf{Networking Topology} - There are multiple topologies available to network devices. The hub-spoke topology and mesh topology are the most relevant to wireless devices.
	\item \textbf{Consensus Algorithm} - Central to our project, we have a variety of consensus algorithms to select from. We narrow our scope to look specifically at distributed consensus algorithms. 
	\item \textbf{MCU} - IoT devices come in multiple different capabilities. While some are quite powerful and packed with significant resources, others are very simple systems meant for trivial tasks. The choice of the IoT devices is important for performance, power consumption, and memory considerations.
	\item \textbf{Battery} - For our prototype, we want to have access to mobile nodes and will thus develop our own prototype board powered by batteries so that it is not tethered to a power source.
\end{itemize}


\begin{table}[ht]
    \scriptsize
    
    % Set row height
    \renewcommand{\arraystretch}{1.3}

    \captionof{table}{Morphological Chart}
    \label{tab:morph_chart}
    
    \begin{center}
        \begin{tabular}{|l|l|l|l|l|l|}
        \hline
        \multicolumn{6}{|c|}{\multirow{2}{*}{\textbf{ramen: Design and Development of a Raft Consensus Algorithm Coupled With a IEEE 802.11 Based Mesh Network for Embedded Systems}}} \\
        \multicolumn{6}{|c|}{}                                                                                   \\ \hline
        \multicolumn{1}{|c|}{Sub-Problem} & \multicolumn{5}{c|}{Available Options}                               \\ \hline
        Networking Protocol &
          IEEE 802.11 &
          IEEE 802.15.4 (Zigbee) &
          IEEE 802.15.3 (UWB) &
          802.15.1 (Bluetooth) &
          LoRaWAN \\ \hline
        Networking Topology               & Hub-Spoke Topology & Mesh Topology & Ring Topology & Bus Topology &  \\ \hline
        Consensus Algorithm               & Paxos              & Raft          & ZAB           & Mencius      &  \\ \hline
        MCU                               & RaspberryPi        & BeagleBone    & ESP8266       & Arduino      &  \\ \hline
        Battery                           & LiPo               & NiMh          & Zinc-Carbon   &              &  \\ \hline
        \end{tabular}
\end{center}
\end{table}
\FloatBarrier


\subsection{Concept Selection}

We also constructed Pugh Charts to aid in the decision-making process when evaluating alternatives compared to a baseline.

\begin{table}[!h]
    \scriptsize
    
    % Set row height
    \renewcommand{\arraystretch}{1.3}

    \captionof{table}{Pugh Chart for Consensus Algorithm Selection with Raft as Base}
    \label{tab:pugh_raft}

    \begin{center}
        \begin{tabular}{@{}*{6}{|p{0.14\textwidth}|@{}}}
        \hline
        \multicolumn{1}{|c|}{Consensus Algorithm} & Ease of Access & Documentation & Relevance & Performance & Sum \\ \hline
        Raft    & Base & Base & Base & Base &    \\ \hline
        Paxos   & -1   & 0    & -1   & 0    & -2 \\ \hline
        Mencius & -1   & -1   & -1   & 0    & -3 \\ \hline
        ZAB     & -1   & -1   & -1   & -1   & -4 \\ \hline
        \end{tabular}
    \end{center}
\end{table}
\FloatBarrier

\begin{table}[!h]
    \scriptsize
    
    % Set row height
    \renewcommand{\arraystretch}{1.3}

    \captionof{table}{Pugh Chart for Consensus Algorithm Selection with Paxos as Base}
    \label{tab:pugh_paxos}
    
    \begin{center}
        \begin{tabular}{@{}*{6}{|p{0.14\textwidth}|@{}}}
        \hline
        \multicolumn{1}{|c|}{Consensus Algorithm} & Ease of Access & Documentation & Relevance & Performance & Sum \\ \hline
        Raft    & 1    & 1    & 0    & 0    & 2  \\ \hline
        Paxos   & Base & Base & Base & Base &    \\ \hline
        Mencius & 0    & -1   & -1   & 0    & -2 \\ \hline
        ZAB     & 0    & -1   & -1   & -1   & -3 \\ \hline
        \end{tabular}
    \end{center}
\end{table}
\FloatBarrier

\begin{table}[!h]
    \scriptsize
    
    % Set row height
    \renewcommand{\arraystretch}{1.3}

    \captionof{table}{Pugh Chart for Consensus Algorithm Selection with Mencius as Base}
    \label{tab:pugh_mencius}
    
    \begin{center}
        \begin{tabular}{@{}*{6}{|p{0.14\textwidth}|@{}}}
        \hline
        \multicolumn{1}{|c|}{Consensus Algorithm} & Ease of Access & Documentation & Relevance & Performance & Sum \\ \hline
        Raft    & 1    & 1    & 1    & 0    & 3  \\ \hline
        Paxos   & 1    & 1    & 0    & 0    & 2  \\ \hline
        Mencius & Base & Base & Base & Base &    \\ \hline
        ZAB     & 0    & 0    & 0    & -1   & -1 \\ \hline
        \end{tabular}
    \end{center}
\end{table}
\FloatBarrier

\begin{table}[!h]
    \scriptsize
    
    % Set row height
    \renewcommand{\arraystretch}{1.3}

    \captionof{table}{Pugh Chart for Consensus Algorithm Selection with ZAB as Base}
    \label{tab:pugh_zab}
    
    \begin{center}
        \begin{tabular}{@{}*{6}{|p{0.14\textwidth}|@{}}}
        \hline
        \multicolumn{1}{|c|}{Consensus Algorithm} & Ease of Access & Documentation & Relevance & Performance & Sum \\ \hline
        Raft    & 1    & 1    & 1    & 1    & 4 \\ \hline
        Paxos   & 1    & 1    & 1    & 0    & 3 \\ \hline
        Mencius & 0    & 0    & 0    & 1    & 1 \\ \hline
        ZAB     & Base & Base & Base & Base &   \\ \hline
        \end{tabular}
    \end{center}
\end{table}
\FloatBarrier

After 4 iterations of the decision matrix, it is quite clear that the Raft consensus algorithm is the superior choice. Given its superior ease of access and documentation and suitable performance, it will make development relatively easier.

\begin{table}[!h]
    \scriptsize
    
    % Set row height
    \renewcommand{\arraystretch}{1.3}

    \captionof{table}{Pugh Chart for MCU Selection with RaspberryPi as Base}
    \label{tab:pugh_zab}
    
    \begin{center}
        \begin{tabular}{@{}*{7}{|p{0.11\textwidth}|@{}}}
        \hline
        Microcontroller &
          \multicolumn{1}{c|}{Ease of Access} &
          \multicolumn{1}{c|}{Documentation} &
          \multicolumn{1}{c|}{Resource Reasonability} &
          \multicolumn{1}{c|}{Familiarity} &
          \multicolumn{1}{c|}{Mesh Support} &
          \multicolumn{1}{c|}{Sum} \\ \hline
        RaspberryPi & Base & Base & Base & Base & Base &    \\ \hline
        BeagleBone  & -1   & 0    & 0    & -1   & 0    & -2 \\ \hline
        ESP8266     & 0    & 0    & 1    & 0    & 0    & 1  \\ \hline
        Arduino     & 0    & 0    & -1   & 0    & -1   & -2 \\ \hline
        \end{tabular}
    \end{center}
\end{table}
\FloatBarrier

\begin{table}[!h]
    \scriptsize
    
    % Set row height
    \renewcommand{\arraystretch}{1.3}

    \captionof{table}{Pugh Chart for MCU Selection with BeagleBone as Base}
    \label{tab:pugh_zab}
    
    \begin{center}
        \begin{tabular}{@{}*{7}{|p{0.11\textwidth}|@{}}}
        \hline
        Microcontroller &
          \multicolumn{1}{c|}{Ease of Access} &
          \multicolumn{1}{c|}{Documentation} &
          \multicolumn{1}{c|}{Resource Reasonability} &
          \multicolumn{1}{c|}{Familiarity} &
          \multicolumn{1}{c|}{Mesh Support} &
          \multicolumn{1}{c|}{Sum} \\ \hline
        RaspberryPi & 1    & 0    & 0    & 1    & 0    & 2 \\ \hline
        BeagleBone  & Base & Base & Base & Base & Base &   \\ \hline
        ESP8266     & 1    & 0    & 1    & 1    & 0    & 3 \\ \hline
        Arduino     & 1    & 0    & 0    & 1    & -1   & 1 \\ \hline
        \end{tabular}
    \end{center}
\end{table}
\FloatBarrier

\begin{table}[!h]
    \scriptsize
    
    % Set row height
    \renewcommand{\arraystretch}{1.3}

    \captionof{table}{Pugh Chart for MCU Selection with ESP8266 as Base}
    \label{tab:pugh_zab}
    
    \begin{center}
        \begin{tabular}{@{}*{7}{|p{0.11\textwidth}|@{}}}
        \hline
        Microcontroller &
          \multicolumn{1}{c|}{Ease of Access} &
          \multicolumn{1}{c|}{Documentation} &
          \multicolumn{1}{c|}{Resource Reasonability} &
          \multicolumn{1}{c|}{Familiarity} &
          \multicolumn{1}{c|}{Mesh Support} &
          \multicolumn{1}{c|}{Sum} \\ \hline
        RaspberryPi & 0    & 0    & -1   & 0    & 0    & -1 \\ \hline
        BeagleBone  & -1   & 0    & -1   & -1   & 0    & -3 \\ \hline
        ESP8266     & Base & Base & Base & Base & Base &    \\ \hline
        Arduino     & 0    & 0    & -1   & 0    & -1   & -2 \\ \hline
        \end{tabular}
    \end{center}
\end{table}
\FloatBarrier

\begin{table}[!h]
    \scriptsize
    
    % Set row height
    \renewcommand{\arraystretch}{1.3}

    \captionof{table}{Pugh Chart for MCU Selection with Arduino as Base}
    \label{tab:pugh_zab}
    
    \begin{center}
        \begin{tabular}{@{}*{7}{|p{0.11\textwidth}|@{}}}
        \hline
        Microcontroller &
          \multicolumn{1}{c|}{Ease of Access} &
          \multicolumn{1}{c|}{Documentation} &
          \multicolumn{1}{c|}{Resource Reasonability} &
          \multicolumn{1}{c|}{Familiarity} &
          \multicolumn{1}{c|}{Mesh Support} &
          \multicolumn{1}{c|}{Sum} \\ \hline
        RaspberryPi & 0    & 0    & 0    & 0    & 1    & 1  \\ \hline
        BeagleBone  & -1   & 0    & 0    & -1   & 1    & -1 \\ \hline
        ESP8266     & 0    & 0    & 1    & 0    & 1    & 2  \\ \hline
        Arduino     & Base & Base & Base & Base & Base &    \\ \hline
        \end{tabular}
    \end{center}
\end{table}
\FloatBarrier

Having completed four iterations of the decision matrix for MCU selection, we are confident in our choice of the ESP8266, primarily due to its mid-tier on-board resources, which allow it to be more versatile in its applications.